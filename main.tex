\documentclass[a4paper,man,natbib]{apa6}

\usepackage[english]{babel}
\usepackage[utf8x]{inputenc}
\usepackage{amsmath}
\usepackage{graphicx}
\usepackage[colorinlistoftodos]{todonotes}

\title{INPUTS, PROCESSES AND THE USE OF GROUPWARE: KEYS TO UNDERSTAND RESULTS OF WORK GROUPS IN UNIVERSITY STUDENTS}
\shorttitle{INPUTS, PROCESSES AND THE USE OF GROUPWARE}
\author{Contreras, Lourdes, Pulido-Martos, Manuel, and Cano, M. Carmen}
\affiliation{University of Jaén (SPAIN)}

\abstract{The models of efficacy of work groups propose two mechanisms of action regarding the results of groups: direct effects of work groups’ resources (inputs) and indirect effects of these resources, acting on the results of groups through group processes. These models also highlight the mediator/moderator role of some variables of the organizational context, as for example the use of technologies to support the work of the group, namely groupware, among others. Results of groups include the perception of members about them (satisfaction with both the results and the group performance). As part of a teaching innovation project, two objectives were established. First, an intervention aimed to increase the use of groupware in practical exercises of work groups was designed, expecting its influence on the perception of results. This intervention provided information about the groupware and different resources to easily access to these types of technologies. Second, this work was aimed to analyse the role of different inputs and group processes on the perception of results achieved by work groups in practical exercises. The sample included 293 Psychology students from the University of Jaén (Spain). Participants completed several assessment instruments in two different moments (time 1 and time 2): before and after the implementation of the program and the fulfilment of practical exercises (10 weeks spacing). The final sample consisted of 247 students, who completed the two assessments. Regarding the first aim, along with the information about the use and frequency of groupware, it was also included the perception of the group members about the results of work group, which was assessed by self-report measures. By using different variance analyses, results showed that, after the intervention program, participants increased the length of use of groupware, although they did not increased the frequency of use. Otherwise, no significant data were found regarding the perception of results of the work group. With respect to the first objective, results indicate that the simple use of informative strategies could be enough to increase the use of technologies to support the work in groups, although this is not enough to explain the variations in the perception of results of groups. Regarding the second objective, and by using self-report measures, information about the inputs used and group processes (tasks design, group design, tasks development, individual attention and conflicts management) was evaluated. Given that the intervention designed did not influence the perception of results, the variable groupware was not included in the following analyses. A multiple linear regression analysis was performed, including as criterion variables the perception of results achieved by the groups in time 2. Furthermore, as predictor variables were included the perception of results obtained by the groups in time 1 (in the first step), the inputs and group processes in time 1 (in the second step), and the inputs and group processes in time 2 (in the third step). Results revealed that the inputs and groups processes in time 2 added an additional percentage of variance to explain the results perceived in time 2, in addition to the contribution of results perceived, inputs and group processes in time 1. These data highlight the importance of models of efficacy of work groups to understand the functioning and results of work groups within educational contexts.}

\begin{document}
\maketitle

\section{Introduction}

Work groups have become a contemporary reality with a great significance in the current society1. Their relevance, both in work and educative contexts, has lead to the development of a research body, which is focused on the identification and analysis of the factors that contribute to the achievement of results within these work groups. Furthermore, the competences related to the work in groups, as well as the acquisition and development of these competences, is a strategic objective for the future research, due to the incorporation of professionals in work groups in different contexts, who will have to interact with people from different places and formation.

Work groups requires the interdependency of their members in relation with the task they are doing, their actions are guided by their common aims and furthermore, and furthermore, they continuously use coordination and mutual adjustment mechanisms. Different models explaining the achievement of results in work groups have been developed, and in this regard, multiple review papers have analysed their main characteristics. Many of these models are grouped under the perspective “input-process-output” (IPO) that, although not without criticism, has become a consolidated approach to explain the results of work groups.



\subsection{Methodology}

\subsection{Participants}

As part of a teaching innovation project, students of third and fourth course of the Psychology Degree of the University of Jaén (Spain) were offered to be participants of an intervention program. A total of 293 students completed several assessment instruments in two different moments (time 1 and time 2). The final sample consisted of 247 students, who completed the two assessments.

\subsection{Instruments}

An ad hoc instrument to assess factors associated to the functioning of work groups was used17. It includes three scales: “antecedents”, “processes” and “results”, with a total of 23 items with a likert scale (1 = totally disagree; 5= totally agree). The first scale (antecedents) assesses the factor “task and group design” (6 items, alpha coefficient = .80). The second scale (processes) assesses the factors “tasks development” (5 items) and “individual attention and conflict management” (4 items), with alpha coefficient = .90. The third scale (results) assesses the factors “production and satisfaction” (4 items) and “group atmosphere” (3 items), with alpha coefficient = .94. Regarding the scale “results, in the current study the items related to the factor “production and satisfaction” were used.

Other aspects related to the use of groupware were also evaluated: instant messaging, video calls, email, phone calls, graphs and text presentations, information from databases, repositories and teaching platforms, instruments and communication apps to contact with relevant persons to carry out the work, instruments and apps to do the work within the group (e.g. projects managements, agenda creation, decision making, etc.). Concretely, the average length of daily use of technologies for the work group and the average frequency of daily use were evaluated. In case of positive answers about the use of these technologies, participants were asked to indicated, for each type of technology, the average length of daily use with a likert scale (1= occasionally, 2= less than 30 minutes, 3= between 30 minutes and 1 hour, 4= between 1 and 2 hours, 5= between 2 and three hours, 6= between 3 and 4 hours, 7= more than 4 hours). Participants were also asked to respond about the average frequency of daily use by using a likert scale (1= occasionally, 2= once in a day, 3= between 2 and four times, 4=between 4 and 6 times, 5= between 6 and 8 times, 6=, between 8 and 10 times, = more than 10 times). 


\subsection{Comments}

You can add inline TODO comments with the todonotes package, like this:
\todo[inline, color=green!40]{This is an inline comment.}

\subsection{References}

LaTeX automatically generates a bibliography in the APA style from your .bib file. The citep command generates a formatted citation in parentheses \citep{Lamport1986}. The cite command generates one without parentheses. LaTeX was first discovered by \cite{Lamport1986}.

\subsection{Tables and Figures}

Use the table and tabular commands for basic tables --- see Table~\ref{tab:widgets}, for example. You can upload a figure (JPEG, PNG or PDF) using the files menu. To include it in your document, use the includegraphics command as in the code for Figure~\ref{fig:frog} below.

% Commands to include a figure:
\begin{figure}
\centering
\includegraphics[width=0.5\textwidth]{Proyecto.jpg}
\caption{\label{fig:frog}Figure of Proyect.}
\end{figure}

\begin{table}
\centering
\begin{tabular}{l|r}
Item & Quantity \\\hline
Widgets & 42 \\
Gadgets & 13
\end{tabular}
\caption{\label{tab:widgets}An example table.}
\end{table}

\subsection{Mathematics}

\LaTeX{} is great at typesetting mathematics. Let $X_1, X_2, \ldots, X_n$ be a sequence of independent and identically distributed random variables with $\text{E}[X_i] = \mu$ and $\text{Var}[X_i] = \sigma^2 < \infty$, and let
$$S_n = \frac{X_1 + X_2 + \cdots + X_n}{n}
      = \frac{1}{n}\sum_{i}^{n} X_i$$
denote their mean. Then as $n$ approaches infinity, the random variables $\sqrt{n}(S_n - \mu)$ converge in distribution to a normal $\mathcal{N}(0, \sigma^2)$.

\subsection{Lists}

You can make lists with automatic numbering \dots

\begin{enumerate}
\item Like this,
\item and like this.
\end{enumerate}
\dots or bullet points \dots
\begin{itemize}
\item Like this,
\item and like this.
\end{itemize}

We hope you find write\LaTeX\ useful, and please let us know if you have any feedback using the help menu above.

\bibliography{example}

\end{document}

%
% Please see the package documentation for more information
% on the APA6 document class:
%
% http://www.ctan.org/pkg/apa6
%